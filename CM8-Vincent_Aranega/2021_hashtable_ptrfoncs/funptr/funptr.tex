\documentclass{beamer}

\usepackage[frenchb]{babel}
\usepackage[T1]{fontenc}
\usepackage[utf8]{inputenc}

\AtBeginSection[]
{
   \begin{frame}
       \frametitle{Outline}
       \tableofcontents[
       currentsubsection,
       sectionstyle=show/shaded,
       subsectionstyle=hide,
       ]
   \end{frame}
}

\usetheme{Rochester}
\setbeamertemplate{footline}[frame number]

\title[Short Paper Title]{Pointeur de fonctions}
%\subtitle{Programmation Avancée}
\author{Vincent Aranega\\%
\texttt{vincent.aranega@genmymodel.com}\\%
Pompé sur le cours de G. Bianchi, Université de Bordeaux}
\date{\today}
\beamertemplatenavigationsymbolsempty{}

\begin{document}

  \maketitle

  \begin{frame}{Rappel}
    \begin{itemize}
      \item{Variable $\rightarrow$ stockée soit dans la pile, soit dans le segment de données (\texttt{.data} ou \texttt{.bss})}
      \item{Une variable à une adresse qui peut être stockée dans un pointeur}
      \item{Un pointeur = (adresse mémoire + type cible)}
      \begin{itemize}
        \item{\texttt{int *p\_int;}}
        \item{\texttt{float *p\_float;}}
        \item{\texttt{char * p\_char;}}
        \item{\texttt{double *p\_double;}}
      \end{itemize}
    \end{itemize}
  \end{frame}

  \begin{frame}{Rappel}
    \begin{itemize}
      \item{Le code d'un programme est stocké dans une zone mémoire : le segment de code (\texttt{.text})}
      \item{Toute fonction, comme tout autre objet du programme, a donc également une adresse mémoire}
      \item{Afin de manipuler cette dernière, il faudra avoir connaissancedu ``type'' de la fonction (à l’instar des variables)}
    \end{itemize}
  \end{frame}

  \begin{frame}{Récupération de l'adresse d'une fonction}
    \begin{minipage}{.5\textwidth}
      TEXT 1
    \end{minipage}% This must go next to `\end{minipage}`
    \begin{minipage}{.5\textwidth}
      TEXT 2
    \end{minipage}
  \end{frame}

\end{document}

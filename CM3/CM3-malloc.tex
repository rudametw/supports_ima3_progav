%\documentclass[12pt,serif]{beamer}
%\documentclass[tikz,12pt,svgnames]{beamer}
\documentclass[table,handout,tikz,12pt,svgnames]{beamer}
\usepackage{CM-preamble}
\subtitle{\Huge Allocation Dynamique}
%\date{8 février 2016}
\date{CM3}

\begin{document}

\begin{frame}
	\titlepage
\end{frame}

\begin{frame}[fragile=singleslide]
	\frametitle{Allocation de mémoire}
		\begin{block}{Variables automatiques} %Gestion dynamique de la mémoire}
			\begin{itemize}
                \item Variables de bloc, paramètress de fonctions
			\item Crées automatiquement à l'exécution
            \item Allocation dynamique sur la pile (stack)
			\end{itemize}
		\end{block}
		\begin{block}{Variables dynamiques} %Gestion dynamique de la mémoire}
			\begin{itemize}
			\item Créées et détruites dynamiquement et explicitement
			\item Fonctions \texttt{malloc} et \texttt{free}
			\item Allocation sur le tas (heap)
			\end{itemize}
		\end{block}
\end{frame}

\begin{frame}[fragile=singleslide]
	\frametitle{Erreur d'allocation }
\begin{minted}[mathescape=true,escapeinside=||,tabsize=4]{c}
/* À ne pas faire */	

int * allouer_entier() {
	int var_static ; // alloué sur la pile
	printf("var_static address is : %p\n",
							  &var_static);
	return &var_static ;
    /* var_static est libéré lors
    de la fin de la fonction */
}			
\end{minted}
\end{frame}

\begin{frame}[fragile=singleslide]
	\frametitle{Allocation dynamique --- malloc}
		\begin{block}{Fonction \texttt{malloc}} %Gestion dynamique de la mémoire}
			\begin{itemize}
				\item \texttt{void * malloc (size\_t taille);}
				\begin{itemize}
					\item Alloue dynamiquement dans le tas un espace de \texttt{taille} octets
					\item Résultat : \textit{pointeur non typé} vers la zone allouée
					\item Pointeur peut être converti automatiquement vers le type désiré (conversion implicite)
					\item Besoin de \texttt{\#include<stdlib.h>}
				\end{itemize}
			\end{itemize}
		\end{block}
\end{frame}

\begin{frame}[fragile=singleslide]
	\frametitle{Allocation dynamique --- Exemples}
		\begin{block}{Allocation dynamique d'un entier} %Gestion dynamique de la mémoire}
			\begin{minted}[mathescape=true,escapeinside=||,tabsize=4]{c}
	int *pt;
	//pt = (int *) malloc(sizeof(int));
	pt = malloc(sizeof(int));
	*pt = 42; //utilisation
			\end{minted}
		\end{block}
		\begin{block}{Allocation dynamique d'un tableau d'entiers}
%			\begin{itemize}
			\begin{minted}[mathescape=true,escapeinside=||,tabsize=4]{c}
	int n; int *pt;
	scanf("%d", &n);
	pt = (int *) malloc(n*sizeof(int)); 
	//Accès
	*pt = 11;          //premier élément
	*(pt+1) = 22 ;     //deuxième
	pt[2] = 33 ;       //troisième
	*(pt+n-1) = 9876 ; //dernier
			\end{minted}
		\end{block}
\end{frame}

%\begin{frame}[fragile=singleslide]
%	\frametitle{Allocation dynamique --- Structures}
%		\begin{block}{} %Gestion dynamique de la mémoire}
%			\begin{minted}[mathescape=true,escapeinside=||,tabsize=4]{c}	
%typedef struct {
%	int j,m,a;
%} Date;
%
%Date *pDate= (Date *) malloc(sizeof(Date));
%
%/*ex. utilisation :*/
%scanf("%d%d%d"&(pDate->j),
%				   &(pDate->m),
%				   &(pDate->a));
%			\end{minted}
%		\end{block}
%\end{frame}

\begin{frame}[fragile=singleslide]
	\frametitle{Allocation dynamique --- Structures}
		\begin{block}{} %Gestion dynamique de la mémoire}
		\vspace{-0.5cm}
		\inputminted[
%		frame=lines,
%		framesep=2.5mm,
		baselinestretch=0.7,
		fontsize=\footnotesize,
		linenos,
		tabsize=4,
%		bgcolor=light-gray
		]
		{c}{../codes/struct_declare.c}
		\end{block}
\end{frame}

%\begin{frame}[fragile=singleslide]
%	\frametitle{Allocation dynamique --- Structures}
%		\begin{block}{Tableau de structures} %Gestion dynamique de la mémoire}
%			\begin{minted}[mathescape=true,escapeinside=||,tabsize=2]{c}
%	int n; Date *pt; /*variables*/
%	scanf("%d", &n); /*taille du tableau*/
%	
%	/*pt = (Date *) malloc( n * sizeof(Date));*/
%	pt = malloc( n * sizeof *pt);
%	
%	/*utilisation:*/
%	scanf("%d%d%d", &(pt[0].j), &pt[0].m, &pt[0].a);
%
%	printf("Date %d/%d/%d\n", pt[0].j,
%													  pt[0].m,
%													  pt[0].a);
%	free(pt); pt = NULL;
%			\end{minted}
%		\end{block}
%\end{frame}

\begin{frame}[fragile=singleslide]
	\frametitle{Allocation dynamique --- Structures}
		\begin{block}{Tableau de structures} %Gestion dynamique de la mémoire}
%		\vspace{-0.5cm}
		\inputminted[
%		frame=lines,
%		framesep=2.5mm,
%		baselinestretch=0.8,
		fontsize=\small,
%		linenos,
		tabsize=2,
		firstline=16,lastline=27
%		bgcolor=light-gray
		]
		{c}{../codes/struct_declare_array.c}
		\end{block}
\end{frame}

\begin{frame}[fragile=singleslide]
	\frametitle{Allocation dynamique --- Liste contiguë}
%		\begin{block}{}
%		\vspace{-0.5cm}
		\inputminted[
%		frame=lines,
%		framesep=2.5mm,
%		baselinestretch=0.8,
		fontsize=\small,
%		linenos,
		tabsize=2,
		firstline=14,lastline=28
%		bgcolor=light-gray
		]
		{c}{../codes/alloc_listes_double_pointer.c}
%		\end{block}
\end{frame}

\begin{frame}[fragile=singleslide]
	\frametitle{Allocation dynamique --- Liste contiguë}
%		\begin{block}{}
%		\vspace{-0.5cm}
		\inputminted[
%		frame=lines,
%		framesep=2.5mm,
%		baselinestretch=0.8,
		fontsize=\small,
%		linenos,
		tabsize=2,
		firstline=44,lastline=58
%		bgcolor=light-gray
		]
		{c}{../codes/alloc_listes_double_pointer.c}
%		\end{block}
\end{frame}

\begin{frame}[fragile=singleslide]
	\frametitle{Fonction \texttt{free}}
	\begin{block}{} %Gestion dynamique de la mémoire}
		\begin{itemize}
			\item \texttt{void free(void *ptr);}
			\begin{itemize}
				\item libère l'espace mémoire pointé par \texttt{ptr} (précédemment 
				alloué)
			\end{itemize}
		\end{itemize}
	\end{block}
	\begin{block}{}
		\begin{itemize}
			\item Exemple d'utilisation:\\\textit{Suppression du dernier élément de la liste}
		\end{itemize}
		\begin{minted}[mathescape=true,escapeinside=||,tabsize=4]{c}
		free(l.espace[l.dernier]);
		l.dernier -= 1;
		\end{minted}
	\end{block}
\end{frame}

\begin{frame}[fragile=singleslide]
	\frametitle{Listes chaînées --- Implantation en C}
%		\begin{block}{}
%		\vspace{-0.5cm}
		\inputminted[
%		frame=lines,
%		framesep=2.5mm,
%		baselinestretch=0.8,
		fontsize=\small,
%		linenos,
		tabsize=2,
		firstline=4,lastline=10
%		bgcolor=light-gray
		]
		{c}{../codes/alloc_implantation.c}
%		\end{block}
	\vspace{-0.9cm} \begin{center} \noindent\makebox[\linewidth]{\line(2,0){500}} \end{center}  \vspace{-0.7cm}
		\inputminted[
%		frame=lines,
%		framesep=2.5mm,
%		baselinestretch=0.8,
		fontsize=\small,
%		linenos,
		tabsize=2,
		firstline=13,lastline=14
		]
		{c}{../codes/alloc_implantation.c}
	\vspace{-0.9cm} \begin{center} \noindent\makebox[\linewidth]{\line(2,0){500}} \end{center}  \vspace{-0.7cm}
		\inputminted[
%		frame=lines,
%		framesep=2.5mm,
%		baselinestretch=0.8,
		fontsize=\small,
%		linenos,
		tabsize=2,
		firstline=16,lastline=20
		]
		{c}{../codes/alloc_implantation.c}
\end{frame}

\begin{frame}[fragile=singleslide]
	\frametitle{Listes chaînées --- Implantation en C}
%		\begin{block}{}
%		\vspace{-0.5cm}
		\inputminted[
%		frame=lines,
%		framesep=2.5mm,
%		baselinestretch=0.8,
		fontsize=\small,
%		linenos,
		tabsize=2,
		firstline=29,lastline=32
%		bgcolor=light-gray
		]
		{c}{../codes/alloc_implantation.c}
%		\end{block}
		\begin{center} \noindent\makebox[\linewidth]{\line(2,0){500}} \end{center} 
		\inputminted[
%		frame=lines,
%		framesep=2.5mm,
%		baselinestretch=0.8,
		fontsize=\small,
%		linenos,
		tabsize=2,
		firstline=35,lastline=39
		]
		{c}{../codes/alloc_implantation.c}
+\end{frame}

\begin{frame}[fragile=singleslide]
	\frametitle{Listes chaînées --- Recherche d'un élément}
		\inputminted[
%		frame=lines,
%		framesep=2.5mm,
%		baselinestretch=0.8,
%		fontsize=\small,
%		linenos,
		tabsize=4,
		firstline=44,lastline=54
		]
		{c}{../codes/alloc_implantation.c}
\end{frame}

\begin{frame}[fragile=singleslide]
	\frametitle{Listes chaînées --- Exemple: ajout en tête}
		\inputminted[
%		frame=lines,
%		framesep=2.5mm,
%		baselinestretch=0.8,
%		fontsize=\small,
%		linenos,
		tabsize=4,
		firstline=57,lastline=73
		]
		{c}{../codes/alloc_implantation.c}
\end{frame}
%
%\begin{frame}
%\huge \center END
%\end{frame}


% % % % % % % % % % % % % % % % % % % % % % % % % % %
% END
% % % % % % % % % % % % % % % % % % % % % % % % % % %
\end{document}


% % % % % % % % % % % % % % % % % % % % % % % % % % %
% END
% % % % % % % % % % % % % % % % % % % % % % % % % % %

	\vspace{-0.9cm} \begin{center} \noindent\makebox[\linewidth]{\line(2,0){500}} \end{center}  \vspace{-0.7cm}
	\vspace{-0.9cm} \begin{center} \noindent\makebox[\linewidth]{\rule{\paperwidth}{1.5pt}}  \end{center}  \vspace{-0.7cm}	
